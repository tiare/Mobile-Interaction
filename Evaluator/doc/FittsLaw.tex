%%%%%%%%%%%%%%%%%%%%%%%%%%%%%%%%%%%%%%%%%%%%%%%%%%%%%%%%%%%%%
%% HEADER
%%%%%%%%%%%%%%%%%%%%%%%%%%%%%%%%%%%%%%%%%%%%%%%%%%%%%%%%%%%%%
\documentclass[a4paper,twoside,10pt]{article}
% Alternative Options:
%	Paper Size: a4paper / a5paper / b5paper / letterpaper / legalpaper / executivepaper
% Duplex: oneside / twoside
% Base Font Size: 10pt / 11pt / 12pt

%% Language %%%%%%%%%%%%%%%%%%%%%%%%%%%%%%%%%%%%%%%%%%%%%%%%%
\usepackage[USenglish]{babel} %francais, polish, spanish, ...
\usepackage[T1]{fontenc}

%\usepackage[ansinew]{inputenc}
\usepackage[utf8]{inputenc}	%supports Umlaute
\usepackage{german, ngerman}
\usepackage{eurosym}
\usepackage{color}

\usepackage{lmodern} %Type1-font for non-english texts and characters

%% Packages for Graphics & Figures %%%%%%%%%%%%%%%%%%%%%%%%%%
\usepackage{graphicx} %%For loading graphic files
%\usepackage{subfig} %%Subfigures inside a figure
%\usepackage{tikz} %%Generate vector graphics from within LaTeX

%% Math Packages %%%%%%%%%%%%%%%%%%%%%%%%%%%%%%%%%%%%%%%%%%%%
\usepackage{amsmath}
\usepackage{amsthm}
\usepackage{amsfonts}

%% Other Packages %%%%%%%%%%%%%%%%%%%%%%%%%%%%%%%%%%%%%%%%%%%
\usepackage{a4wide} %%Smaller margins = more text per page.
\usepackage{fancyhdr} %%Fancy headings
%\usepackage{longtable} %%For tables, that exceed one page

\usepackage[parfill]{parskip} 

%%%%%%%%%%%%%%%%%%%%%%%%%%%%%%%%%%%%%%%%%%%%%%%%%%%%%%%%%%%%%
%% DOCUMENT
%%%%%%%%%%%%%%%%%%%%%%%%%%%%%%%%%%%%%%%%%%%%%%%%%%%%%%%%%%%%%
\begin{document}

%\setlength{\parindent}{0pt} %kein Einzug beim Absatzbegin
\pagestyle{fancyplain}

%\title{Aufgabenblatt 3 - Trading Agent Competition} 
%\author{Tiare Feuchtner, Marcel Karsten}
%\date{Abgabetermin: 11.12.2011} %%If commented, the current date is used.
%\maketitle

\rhead{TU Berlin - MPI, WS2012/13}
\lhead{Tiare Feuchtner, Marcel Karsten}
\renewcommand{\headrulewidth}{0px}
%%%%%%%%%%%%%%%%%%%%%%%%%%%%%%%%%%%%%%%%%%%%%%%%%%%%%%%%%%%%%

\begin{center}
\huge{\textbf{Assignment 3 - Fitts' Law}}
\end{center}
\vspace{.5cm}

\section{Fitts' Law Optimization} 
\subsection{Fitts' Law Examples} 
Driving a car --> hitting brake and accelerator (large pedal, small pedal, great distance)\\
Blindly use keys on keyboard (large enter key, small enter key, large/small spaces)\\
Pop-up menus/advertisement on websites (exit button always small, usually at an edge/bottom --> thus far away, always somewhere else, barely visible), firefox back button, windows menu button

Pointing example where Fits' Law does NOT apply: ??? Wack-A-Mole (Game) - since the game's intention is to make it hard to hit the moles it will try to break with fitts' law, or where design does not permit it --> placing Reply and Delete button close to each other, because they are most often used, may lead to mistakes --> navigation panes with drop down menus cause more cursor movement, but significantly help keeping order and grouping elements to reduce cluttering - although it might impede a fluent workflow --> unlock cellphone by "complicated" swipe is a real stumble stone for fluid workflow, but it is meant to be, for only this way it can actually ensure unintensional unlocking.

\subsection{Keyboard redesign} 
Calculating the average movement time MT of pointing to the keyboard and then pointing to the call button.

$MT_1$ ... average Movement Time (old design) \\
$MT_2$ ... average Movement Time (new design) \\
$W = 5$ ... target width (call button or center of keypad) \\
$D_1 = 35$ ...  target distance (old design) \\
$D_2 = 15$ ... target distance (new design) 

$a$ ... start/stop time of device (intercept) \\
$b$ ... inherent speed of device (slope) \\
$ID$ ... index of difficulty

General formula: \\
$MT = a + b \cdot ID$ \\
$MT = a + b \cdot log_2 (1 + \frac{D}{W})$ 

Calculation: \\
$MT_1 = a + b \cdot log_2 (1 + \frac{35}{5}) = a + b \cdot log_2(8) = a + 3b$ \\
$MT_2 = a + b \cdot log_2 (1 + \frac{15}{5}) = a + b \cdot log_2(4) = a + 2b$ \\
Result: $MT_2 = MT_1 - b$ \\
The movement time difference between the two designs is $b$, which clearly rates the new design better.

\section{Fitts' Law Evaluation}
results of all 3 experiments --> screenshots of the 9 targets\\
Experiment 1: Touchpad --> Screenshot of result\\
Experiment 2: Mouse --> Screenshot of result\\
Experiment 3: Tablet / different User? --> Screenshot of result

Table with results..

%\subsection{Resultate}
%Das Walverine-System vermag besonders am Anfang des Spiels sehr gute Ergebnisse zu erzielen. In den weiteren Biet-Runden fällt Walverine jedoch zurück, da die Algorithmen mit welchen gearbeitet wird, die Umweltfaktoren zu wenig berücksichtig. So sollte beispielsweise berücksichtigt werden dass die Agenten mit ihren Aktionen die Preise beeinflussen 

\vspace{1cm}

%\textbf{Referenz}\\
%Shih-Fen Cheng Evan, Evan Leung, Kevin M. Lochner, Daniel M. Reeves, L. Julian Schvartzman,
%and Michael P. Wellman. Walverine: A walrasian trading agent. In Second International
%Joint Conference on Autonomous Agents and Multi-Agent Systems, pages 465–472, 2002.

\end{document}
